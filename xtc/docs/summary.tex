\documentstyle{article}
\begin{document}
\begin{center}
{\LARGE SUMMARY}
\end{center}
\begin{center}
{\bf\Large Format}
\end{center}
\begin{verbatim}
SPACE <name of space variable>
TIME <name of time variable>
SET <option>=<value>
...
# Comments (anywhere in file, preceded by #)

CVAR <name>=<expression>
...
BDRY <name> <0|1> <zero|noflux|periodic|leaky|dynamic> {a} {b} {c}
# a u_x + b u=c #
# b u_t = a u_x + c #
...
VAR <name>=<number>, <name>=<number>,...
...
PAR <name>=<number>, <name>=<number>,...
...
CPAR <none|normal|pernorm|periodize|readfile> <name>=<expression>
...
WEIGHT <none|normal|readfile> <name>=<expression>
...
FUN <name> <nargs> <expression>
...
CFIX <name>=<expression>
...
FIX <name>=<expression>
...
AUX <name>=<expression>
...
CAUX <name>=<expression>
...
<name>'=<expression>
...
<name>'=<expression>
SET PLOTVAR=<name>
DONE
\end{verbatim}
\newpage
\begin{center}
{\bf \Large Options}
\end{center}
\begin{itemize}
\item EXTEND 0-periodic, 1-zero, 2-even, 3-odd. 
\item NPERIOD {\tt<}int{\tt>} For periodizing kernels.
\item GRID {\tt<}int{\tt>} Spatial grid size.  
\item LENGTH {\tt<}float{\tt>} Length of the domain.
\item FAST {\tt<}1,0{\tt>} 1 Speeds it up.
\item METHOD (0)-Euler (1)-BackEul, (2)-Gear, (3)-ModEul. 
\item DELTA\_T {\tt<}float{\tt>}  The time step. 
\item TFINAL {\tt<}float{\tt>}  The output times.
\item TRANS {\tt<}float{\tt>}  Sets the transient time.
\item NOUT {\tt<}int{\tt>}  Sets the number of outputs.
\item PLOTVAR {\tt<}name{\tt>} 3D plot variable.
\item BUFSIZE {\tt<}int{\tt>} Number of timesteps saved.
\item DTMIN {\tt<}float{\tt>} Minimum timestep fo Gear.
\item DTMAX {\tt<}float{\tt>} Maximum time step for Gear
\item TOLERANCE {\tt<}float{\tt>} Sets the Gear tolerance.
\item EPSILON {\tt<}float{\tt>}  Tolerance for backward Euler.
\item JACUSE {\tt<}int{\tt>}  Number of times to reuse the Jacobian in backward Euler.
\item MAXITER {\tt<}int{\tt>} Maximum iterates in backward Euler.
\item TSTART {\tt<}float{\tt>} Initial time. 
\item MAXDERIV {\tt<}int{\tt>}  Maximum number of spatial derivatives 
\item RGB (0)-red/blue (1)-spectral (2) periodic.
\end{itemize}
\newpage
\begin{center}
{\bf \Large Operators}
\end{center}

\begin{itemize}
\item  {\tt conv(u,v,y0,y1,x,type)}  
\[
 \int_{y0}^{y1}u(x-y)v(y) dy
\]
type={\tt PERIODIC, ZERO, EVEN, and ODD}.
\item {\tt fftconv(w,u,x,type)}
\[
 \int_0^L w(x-y) u(y) dy
\]
type={PERIODIC,ZERO,EVEN}.
\item {\tt weight(w,u,y0,y1,x)} 
\[
\int_{y0}^{y1} w(x,y)u(y)
\]
\item {\tt rweight(w,u,y0,y1,x)} 
\[
\int_{y0}^{y1} w(y,x)u(y)
\]
\item {\tt average(u,v,y0,y1)} 
\[
\int_{y0}^{y1} u(s)v(s)ds
\]
\item {\tt biharm(u,x,type)} 
\[
 u_{xxxx}(x)
\]
type = {\tt  PERIODIC, EVEN, ODD}.
\item {\tt int(z,z1,z2)of(<expr>)}  
\[
\int_{z=z_1}^{z=z_2} \hbox{expr}(z) \, dz
\]
\end{itemize}


\end{document}

